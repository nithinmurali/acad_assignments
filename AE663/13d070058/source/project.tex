% Copyright 2004 by Till Tantau <tantau@users.sourceforge.net>.
%
% In principle, this file can be redistributed and/or modified under
% the terms of the GNU Public License, version 2.
%
% However, this file is supposed to be a template to be modified
% for your own needs. For this reason, if you use this file as a
% template and not specifically distribute it as part of a another
% package/program, I grant the extra permission to freely copy and
% modify this file as you see fit and even to delete this copyright
% notice. 
\documentclass{beamer}
\usepackage{listings}
\usepackage{hyperref}
\usepackage{minted}

% There are many different themes available for Beamer. A comprehensive
% list with examples is given here:
% http://deic.uab.es/~iblanes/beamer_gallery/index_by_theme.html
% You can uncomment the themes below if you would like to use a different
% one:
%\usetheme{AnnArbor}
%\usetheme{Antibes}
%\usetheme{Bergen}
%\usetheme{Berkeley}
%\usetheme{Berlin}
%\usetheme{Boadilla}
%\usetheme{boxes}
%\usetheme{CambridgeUS}
%\usetheme{Copenhagen}
%\usetheme{Darmstadt}
%\usetheme{default}
%\usetheme{Frankfurt}
%\usetheme{Goettingen}
%\usetheme{Hannover}
%\usetheme{Ilmenau}
%\usetheme{JuanLesPins}
%\usetheme{Luebeck}
\usetheme{Madrid}
%\usetheme{Malmoe}
%\usetheme{Marburg}
%\usetheme{Montpellier}
%\usetheme{PaloAlto}
%\usetheme{Pittsburgh}
%\usetheme{Rochester}
%\usetheme{Singapore}
%\usetheme{Szeged}
%\usetheme{Warsaw}

\title{pygsheets}

% A subtitle is optional and this may be deleted
\subtitle{sdes Project 2}

\author{Nithin M}


% Let's get started
\begin{document}

\begin{frame}
  \titlepage
\end{frame}


\begin{frame}{Outline}
  \tableofcontents
  % You might wish to add the option [pausesections]
\end{frame}


% Section and subsections will appear in the presentation overview
% and table of contents.
\section{Project Overview}

\begin{frame}{Project Overview}
  Google has rolled out its new version 4 of Google spreadsheet api, which adds a lot of new options. Even though Google has provided an python library for this api, it rather complicated and unintuitive. So pygsheets is meant to be a library which is rather easy to use and is intutive.
\end{frame}

\section{Features}

\begin{frame}{Features}
  \begin{itemize}
  \item {
     Simple to use
  }
  \item {
    Open spreadsheets using title or key
  }
  \item {
    Gives a spreadsheet like access, With intuitive modes like - spreadsheet, worksheet, cell
  }
  \item {
    Doesn't need network connection for each request , work on a local copy and update once all the changes are done
  }
  \end{itemize}
\end{frame}


\begin{frame}{Example}
    \inputminted{python}{eg.py}
\end{frame}



\section{Evaluation Criteria}

\subsection{Proper use of git}
\begin{frame}{Proper use of git}
  \begin{itemize}
      \item {The project uses git and is hosted in github.}
      \item {Proper commit messages}
      \item{using two branches -  master and staging}
  \end{itemize}
\end{frame}

\subsection{Automated tests and coverage}
\begin{frame}{Automated tests and coverage}
  \begin{itemize}
  
  \item {There will two types of test suits}
  \item {
    A test suite that doesn't query the Google API (Mock test)
    \begin{itemize}
        \item{remarkedly speeds up}
        \item{Avoids error-prone credential setup}
        \item{Enables validation even if Internet access is unavailable.}
    \end{itemize}
  }
  \item {
    A test suite that queries the Google API.
    \begin{itemize}
        \item{Measures the production performance}
        \item{Detects any changes in Google api}
        \item{User will have to provide credentials to run this test/ or maybe provide a shared sheet credentials}
    \end{itemize}
  }
  \end{itemize}
\end{frame}


\subsection{Automation}
\begin{frame}{Automation}
  \begin{itemize}
  
    \item {Makefile - TBD}
    \item {setup.py - already developed}
    \item {py.test/nose setup - TDB}
    \item {travis - ci- TDB}
    \item {pypi - TBD}
  \end{itemize}
\end{frame}


\subsection{Documentation using sphinx }
\begin{frame}{Documentation using sphinx }
  \begin{itemize}
  
    \item {Not setup , but most of the code already documented inline in sphinx style}
  \end{itemize}
\end{frame}

\subsection{Code cleanliness/approach/style  }
\begin{frame}{Code cleanliness/approach/style  }
  \begin{itemize}
  
    \item {The IDE i am using has inbuild PEP8 rules validation , so i am taking care of this as i am writing new code.}
    \item{The library is divided in to client and models.}
      \begin{itemize}
        \item{The Models implement the spreadsheet entities -  spreadsheet which can have one or more worksheet which can have one or more cells.}
        \item{The client handles all the backend communication with the server. The models will communicate with the client}
      \end{itemize}

    \item{Meaningful  custom exceptions}
    
  \end{itemize}
\end{frame}

\subsection{Content: Implementation of planned features.}
\begin{frame}{Content: Implementation of planned features.}
  \begin{itemize}
    \item {Features yet to be Implemented}
        \begin{itemize}
            \item {Batch Operations}
            \item {Improve cell manipulations}
            \item {improve formatting options}
            \item {add support for comments}
            \item {Graph features - inserting graphs, etc}
        \end{itemize}
  \end{itemize}
\end{frame}

% All of the following is optional and typically not needed. 
\appendix
\section<presentation>*{\appendixname}
\subsection<presentation>*{For Further Reading}

\begin{frame}[allowframebreaks]
  \frametitle<presentation>{Refrences}
    
  \begin{thebibliography}{10}
    
  \beamertemplatebookbibitems
  % Start with overview books.

  \bibitem{Google Spreadhseet API}
     Pygsheets
    \newblock {\em \url{https://github.com/nithinmurali/pygsheets}}.

 
    
  \beamertemplatearticlebibitems
  % Followed by interesting articles. Keep the list short. 

  \bibitem{pygsheets}
     Spreadhseet API
    \newblock {\em \url{https://developers.google.com/sheets/}}.
    
    
  \end{thebibliography}
\end{frame}

\end{document}


